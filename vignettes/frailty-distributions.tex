%% LyX 2.1.3 created this file.  For more info, see http://www.lyx.org/.
%% Do not edit unless you really know what you are doing.
\documentclass[english]{article}
\usepackage[T1]{fontenc}
\usepackage[latin9]{inputenc}
\usepackage{babel}
\begin{document}

\section{Frailty distributions}

All frailty distributions have support $\omega\in\left(0,\infty\right)$.
It is usually necessary to place constraints on the parameters to
avoid identifiability problems.


\subsection{Gamma}

Frailty values from a gamma distribution are denoted by $\omega\sim\Gamma(\frac{1}{\theta},\frac{1}{\theta})$,
where $E[\omega]=1$ and $Var[\omega]=\theta$. The frailtyr package
uses a one-parameter gamma distribution with shape and scale both
$\frac{1}{\theta}$ and density given by

\begin{equation}
\Gamma(\omega;\theta)=\frac{\omega^{\frac{1}{\theta}-1}\exp\left(\frac{-\omega}{\theta}\right)}{\theta^{\frac{1}{\theta}}\Gamma(\frac{1}{\theta})}\label{eq:density-gamma}
\end{equation}
The special case when $\theta=0$ is taken to be a degenerate distribution
where $\omega=1$, i.e. there is no hidden frailty in the hazard function.
The partial derivative, with respect to the single parameter $\theta$,
has closed form solution

\begin{equation}
\frac{\partial\Gamma(\omega;\theta)}{\partial\theta}=\frac{\left(\frac{\omega}{\theta}\right)^{\frac{1}{\theta}-1}\exp\left(\frac{-\omega}{\theta}\right)\left\{ \ln\left(\frac{\theta}{\omega}\right)+\psi^{(0)}\left(\frac{1}{\theta}\right)+\omega-1\right\} }{\theta^{3}\Gamma\left(\frac{1}{\theta}\right)}\label{eq:deriv-gamma}
\end{equation}



\subsection{Log-normal}

Frailty values from a log-normal distribution are denoted $\omega\sim LN(\theta)$,
where the log-mean and log-variance of $\omega$ are 0 and $\theta$,
respectively. This distribution has $E[\omega]=\exp\frac{\theta}{2}$
and $Var[\omega]=\exp2\theta-\exp\theta$, with density given by

\begin{equation}
LN(\omega;\theta)=\frac{1}{\omega\sqrt{\theta2\pi}}\exp\left\{ \frac{-\left(\ln\omega\right)^{2}}{2\theta}\right\} \label{eq:density-log-norm}
\end{equation}
Similar to the gamma distribution, the special case of $\theta=0$
indicates $\omega=1$. The partial derivative with respect to $\theta$
is given by

\begin{equation}
\frac{\partial LN(\omega;\theta)}{\partial\theta}=\frac{\ln^{2}\left(\omega\right)\exp\left(\frac{-\ln^{2}\omega}{2\theta}\right)}{2\sqrt{2\pi}\theta^{5/2}\omega}-\frac{\exp\left(\frac{-\ln^{2}\omega}{2\theta}\right)}{2\sqrt{2\pi}\theta^{3/2}\omega}\label{eq:deriv-log-norm}
\end{equation}



\subsection{Positive stable}

The one-parameter positive stable distribution is given by $\omega\sim PS(\alpha)$,
where $\beta=1$, $\mu=0$, $0<\alpha<1$, and $\delta=\alpha$ as
defined in Hougaard \cite{hougaard2000analysis}. The density is given
by

\begin{equation}
PS\left(\omega;\alpha\right)=-\frac{1}{\pi\omega}\sum_{k=1}^{\infty}\frac{\Gamma\left(k\alpha+1\right)}{k!}\left(-\omega^{-\alpha}\right)^{k}\sin\left(\alpha k\pi\right)\label{eq:density-posstab}
\end{equation}
When $\alpha=1$, a degenerate distribution at 1 is obtained and there
is no shared frailty. 


\subsection{Inverse Gaussian}

Frailty values from an inverse Gaussian distribution are denoted by
$\omega\sim IG(\theta)$, where $\frac{1}{\theta}$ is the scale and
the mean is fixed at 1. The density is given by

\begin{equation}
IG\left(\omega;\theta\right)=\left(2\pi\theta\omega^{3}\right)^{-1/2}\exp\left\{ \frac{-\left(\omega-1\right)^{2}}{2\theta\omega}\right\} \label{eq:density-inverse-gaussian}
\end{equation}
where $\theta>0$. The IG has $E[\omega]=1$ and $Var[\omega]=\theta$.
Similar to the gamma and log-normal, $\omega$ is taken to be 1 when
$\theta=0$. The derivative wrt. $\theta$ is given by
\[
\frac{\partial IG\left(\omega;\theta\right)}{\partial\theta}=\frac{\left(\omega-1\right)^{2}\exp\left\{ -\frac{\left(\omega-1\right){}^{2}}{2\theta\omega}\right\} }{2\sqrt{2\pi}\theta^{2}\omega\sqrt{\theta\omega^{3}}}-\frac{\omega^{3}\exp\left\{ -\frac{\left(\omega-1\right)^{2}}{2\theta\omega}\right\} }{2\sqrt{2\pi}\left(\theta\omega^{3}\right){}^{3/2}}
\]



\subsection{Power variance function}

The power variance function (PVF) distribution is denoted $PVF(\alpha,\delta,\theta)$
with density

\[
PVF\left(\omega;\alpha,\delta,\theta\right)=\exp\left(-\theta\omega+\frac{\delta^{\alpha}}{\alpha}\right)\frac{1}{\pi}\sum_{k=1}^{\infty}\frac{\Gamma\left(k\alpha+1\right)}{k!}\left(-\frac{1}{\omega}\right)^{\alpha k+1}\sin\left(\alpha k\pi\right)
\]
where $0<\alpha\le1,\theta\ge0,\delta>0$. To avoid identifiability
problems, we let $\delta=\theta=1$ as in \cite{hanagal2009modeling},
and get the one-parameter PVF density

\begin{equation}
PVF\left(\omega;\alpha\right)=\exp\left(-\omega+\alpha^{-1}\right)\frac{1}{\pi}\sum_{k=1}^{\infty}\frac{\Gamma\left(k\alpha+1\right)}{k!}\left(-\frac{1}{\omega}\right)^{\alpha k+1}\sin\left(\alpha k\pi\right)\label{eq:density-pvf}
\end{equation}
When $\alpha=1$, the degenerate distribution with $\omega=1$ is
obtained. The PVF has $E[\omega]=1$ and $Var[\omega]=\left(1-\alpha\right)$.
\end{document}
